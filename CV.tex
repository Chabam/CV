\documentclass[french]{article}
\usepackage[utf8]{inputenc}
\usepackage[T1]{fontenc}
\usepackage{tabu}
\usepackage{hyperref}
\usepackage[explicit,compact]{titlesec}
\usepackage{xcolor}
\usepackage{parskip}

\titleformat{\section}
	{\normalfont\large\bfseries}
	{}
	{0pt}
	{#1}
	[{\titlerule[1pt]}]

\hypersetup{
	colorlinks,
	allcolors=.,
	urlcolor=blue,
}

\newcommand{\separation}{\multicolumn{1}{c}{} \\}
\pagestyle{empty}

\begin{document}

	\begin{center}
		{\Huge --- Félix Chabot ---}
	\end{center}

	\section{Renseignements personnels}
	\begin{tabu} to \textwidth {rX}
		Courriel: & \href{mailto:fchabot1337@gmail.com}{fchabot1337@gmail.com} \\
		Téléphone: & (579) 488-1254 \\
		Langues: & français et anglais \\
		Répertoire \texttt{git}: & \url{https://github.com/Chabam}
	\end{tabu}
	\section{Expériences de travail}
	\begin{tabu} to \textwidth {r|X}
		\separation{}
		Hiver 2020 & {\large \bfseries Développeur junior} \\
		Été 2019 & {

			\textbf{Diffusion Solution Intégrées}

			{\footnotesize Granby, QC}

			Membre d'une équipe attitrée au plus gros client. Communication (en anglais) directement avec celui-ci.
		} \\

		\separation{}

		Automne 2019 & {
			{\large \bfseries Enseignant}

			\textbf{Cégep de Granby}

			{\footnotesize Granby, QC}

			Préparation de cours, poste annuel, travailler avec des jeunes.
		} \\


		\separation{}
		Automne 2017 & {\large \bfseries Développeur stagiaire} \\
		Été 2018 & {
			\textbf{Diffusion Solution Intégrées}

			{\footnotesize Granby, QC}

			Méthodologie Agile, programmation <<full-stack>>.
		} \\

		\separation{}
		Hiver 2017 & {
			{\large \bfseries Développeur stagiaire}

			\textbf{CYME International Inc.}

			{\footnotesize St-Bruno-de-Montarville, QC}

			Recherche et développement avec des technologies du web.
		} \\

		\separation{}
		Hiver 2016 & {
			{\large \bfseries Développeur stagiaire}

			\textbf{Plastiques Berry Canada Inc.}

			{\footnotesize Waterloo, QC}

			Développement de A à Z d'une application intranet en \textit{MVC.net}
		} \\
	\end{tabu}

	\section{Éducation}
	\begin{tabu} to \textwidth {r|X}
		2019 & {
			{\large \bfseries Baccalauréat en informatique}

			Université de Sherbrooke
		} \\
		\separation{}
		2016 & {
			{\large \bfseries Techniques de l'informatique}

			Cégep de Granby
		} \\
	\end{tabu}

	\section{Compétences}

	\textbf{Compétences générales} \\
	La communication, le travail en équipe, la vulgarisation, bonne gestion du stress, l'apprentissage, la patience.

	\textbf{Langages} \\
	{\ttfamily bash, C, C\#, C++, CSS, HTML, Haskell, Java, JavaScript (ES6), \\ Objective-C, Python, Rust}

	\textbf{Outils} \\
	{\ttfamily AWS, Confluence, Hibernate, IntelliJ, Jira,} \LaTeX{\ttfamily , PyCharm, React, \\ SVN, Spring, Visual Studio, Visual Studio Code, git, node.js, vim}

	\textbf{Base de données} \\
	{\ttfamily MySQL, SQLite, SQL Server}

	\section{Projets}
	{\large \bfseries \href{https://vecpad.netlify.com/}{Vecpad}:}
	{\large un visualiseur d'algèbre linéaire en 3D}

	Réalisé avec \href{https://reactjs.org/}{React} et \href{https://threejs.org/}{three.js}, ce projet a pour but d'aider les élèves des cours des sciences de l'image à mieux se familiariser avec les différents concepts de l'algèbre linéaire à l'aide de visualisations.

	\bigskip
	{\large \bfseries \href{https://github.com/daniel-junior-dube/comfywm}{Comfy}:}
	{\large un gestionnaire de fenêtres par pavage}

	En collaboration avec mon collègue \textit{Daniel-Junior Dubé}, nous avons créé cette composante système à l'aide du langage de programmation \href{https://www.rust-lang.org/}{Rust}. Celle-ci est compilée et fonctionne entièrement sous un environnement \texttt{GNU/Linux} à l'aide du nouveau protocole \href{https://wayland.freedesktop.org/}{Wayland}.

	\bigskip
	{\large \bfseries \href{https://github.com/Chabam/digital-audio-boombox}{Digital-Audio-Boombox}:}
	{\large un lecteur de fichier audio terminal}

	Ce petit projet est une preuve de concept pour démontrer un système à temps réel. L'application communique directement dans un flux audio système à l'aide de la librairie \href{http://www.portaudio.com/}{PortAudio}. Elle est écrite en \texttt{C++17}.

	\section{Intérêts}
	\begin{itemize}
		\item J'adore la musique, j'ai récemment commencé la guitare.
		\item J'essaie de rester en forme à l'aide de l'escalade et l'entraînement physique.
		\item Je me tiens à jour en lisant des livres techniques. J'adore les livres \textit{O'Reilly}!
		\item Comme tout le monde, parfois je m'écrase devant une bonne série en fin de journée!
	\end{itemize}
\end{document}